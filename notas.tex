\documentclass{article}
\usepackage[utf8]{inputenc}
\usepackage[spanish]{babel}
\usepackage{graphicx}
\usepackage{geometry}
\usepackage{multicol}
\usepackage{amsmath}
\usepackage{amsthm,amsfonts,amssymb}%paquetes AMS
\usepackage{hyperref}

\begin{document}

\pagestyle{empty}
\begin{center}
\begin{figure}[h]
\centering


\end{figure}
\Large
\hrule
\vspace{4mm}
\textbf{Notas del Curso de Programación en Go}\\

\vspace{4mm}
\hrule
\large
\vfill
Autor\\

Wilson Eduardo Jerez Hernández \\
\vfill
Profesor\\

Osmandi Gómez
\vfill
Platzi\\
Desarrollo Backend con Go\\
Curso de Programación en Go\\
\end{center}
\newpage


\addtocontents{toc}{\hfill \textbf{Página} \par}
\tableofcontents
\newpage

\section{variables, constantes y zero values}
\subsection{Declaración de constantes} 
const pi float64 = 3.14
const pi2 = 3.14
fmt.Println("pi:", pi)
fmt.Println("pi2:", pi2)
\subsection{Declaración de variables enteras} 
//coloca los dos puntos cuando la variable no a sido creada con anterioridad
base := 12
//otra forma es
var altura int = 14
//otra forma
var area int

fmt.Println(base, altura, area) //se necesita utulizar toda variable que se declara.
\subsection{Zero values} 
//por defecto estos valores tienen.
var a int     //0
var b float64 //0
var c string  //string vacio
var d bool    // false
fmt.Println(a, b, c, d)
\section{Operadores aritméticos}
\section{Tipos de datos primitivos} 
//Numeros enteros
//int = Depende del OS (32 o 64 bits)
//int8 = 8bits = -128 a 127
//int16 = 16bits = $-2^15 a 2^15-1$
//int32 = 32bits = $-2^31 a 2^31-1$
//int64 = 64bits = $-2^63 a 2^63-1$

//Optimizar memoria cuando sabemos que el dato simpre va ser positivo
//uint = Depende del OS (32 o 64 bits)
//uint8 = 8bits = 0 a 127
//uint16 = 16bits = $0 a 2^15-1$
//uint32 = 32bits = $0 a 2^31-1$
//uint64 = 64bits = $0 a 2^63-1$

//numeros decimales
// float32 = 32 bits = $+/- 1.18e^-38 +/- -3.4e^38$
// float64 = 64 bits = $+/- 2.23e^-308 +/- -1.8e^308$

//textos y booleanos
//string = ""
//bool = true or false

//numeros complejos
//Complex64 = Real e Imaginario float32
//Complex128 = Real e Imaginario float64
//Ejemplo : c:=10 + 8i
\end{document}